\documentclass{article}
\usepackage[utf8]{inputenc}
\usepackage[english]{babel}
\usepackage{dirtree} % introduction directory tree
\usepackage{minted} % syntax highlight with minted
\usepackage[most, minted]{tcolorbox}
\usepackage{etoolbox} % box to minted

% myminted env adapted from https://tex.stackexchange.com/questions/396723/minted-line-number-overlapped-by-tcolorbox-frame
\newtcblisting{myminted}{%
    listing engine=minted,
    minted language=prolog,
    listing only,
    breakable,
    enhanced,
    minted options = {
        linenos,% line numbers
        breaklines=true,
        breakbefore=.,
        fontsize=\footnotesize,
        numbersep=2mm
    },
    overlay={% grey box for lineos
        \begin{tcbclipinterior}
            \fill[gray!25] (frame.south west) rectangle ([xshift=4mm]frame.north west);
        \end{tcbclipinterior}
    }
}

% Add colored box to minted env
\BeforeBeginEnvironment{minted}{%
  \begin{tcolorbox}[breakable, enhanced]}%
\AfterEndEnvironment{minted}{\end{tcolorbox}}%

\begin{document}

\Large
\begin{center}
  CA208 Assignment 2, binary trees\\
  \hspace{10pt}

  \large
  Charles PAULET \\
  \hspace{10pt}

  \small
  charles.paulet2@mail.dcu.ie \\
  ECSAOO 18103197 \\
  \bigbreak

  \today
\end{center}
\hspace{10pt}

\normalsize
I confirm that this assignment is my own work, is not copied from any other
person's work. I confirm that I have read and understood the school's policy
on plagiarism.

\newpage

\section*{Document}

\medbreak
\dirtree{%
.1 bt.pl : binary tree prolog sources.
.1 bt.plt : binary tree prolog tests (plunit).
.1 run\_tests.sh : script to start the tests.
.1 delivery.tex : this document source file.
.1 delivery.pdf : this document.
.1 Makefile : script to compile delivery.tex.
}

\section*{Exercices}

  \subsection*{emptyBT}

    An empty binary tree matches \mintinline{prolog}{nil} and the
    \mintinline{prolog}{nil}-root value node \mintinline{prolog}{bTree(nil,nil,nil)}.

    \begin{minted}{prolog}
emptyBT():- !.
emptyBT(nil):- !.
emptyBT(bTree(nil,nil,nil)).
    \end{minted}

  \subsection*{bTree(N,T1,T2)}

    A binary tree is defined as \mintinline{prolog}{bTree(N,T1,T2)} where N is
    the value of the node \mintinline{prolog}{bTree} and where
    \mintinline{prolog}{T1} and  \mintinline{prolog}{T2} are respectively the
    left and the right leaf of the node. \mintinline{prolog}{T1 @=< N} and
    \mintinline{prolog}{T2 @> N}. The \mintinline{prolog}{@} in the operator
    means it applies for the Standard Order of Terms
    \footnote{http://www.swi-prolog.org/pldoc/man?section=standardorder}.\\
    To be valid, I consider that the node value cannot be \mintinline{prolog}{nil}.
    \mintinline{prolog}{T1} and  \mintinline{prolog}{T2} might be subtrees, we
    should match it.

    \begin{minted}{prolog}
bTree(N, nil, nil):- N \= nil.
bTree(N, bTree(LN,LL,LR), nil):- /* left subtree only */
  N \== nil,
  N @>= LN,
  bTree(LN,LL,LR),!.
bTree(N, nil, bTree(RN,RL,RR)):- /* right subtree only */
  N \== nil,
  N @< RN,
  bTree(RN,RL,RR),!.
bTree(N, bTree(LN,LL,LR), bTree(RN,RL,RR)):- /* subtrees */
  N \== nil,
  N @>= LN, N @< RN,
  bTree(LN,LL,LR),
  bTree(RN,RL,RR),!.
    \end{minted}

  \subsection*{inorder(T,L)}

    Inorder traversal is reading data from the bottom left to the bottom right.
    I manage to follow these steps :

    \begin{enumerate}
      \item check is current node is \mintinline{prolog}{nil}.
      \item traverse the left subtree \mintinline{prolog}{T1}.
      \item store the bottom left node value.
      \item traverse the right subtree \mintinline{prolog}{T2}.
      \item store the bottom right node value.
    \end{enumerate}

    In order to respect the rule head \mintinline{prolog}{inorder(T,L)}, I added
    an auxialiary function.

    \begin{minted}{prolog}
inorder(T,L):-inorder_(T,L),!.
inorder_(nil,[]).
inorder_(bTree(N,T1,T2),List):-
  inorder_(T1,L1),
  merge_list(L1,[N],T),
  inorder_(T2,L2),
  merge_list(T,L2,List).
    \end{minted}

  \subsection*{postorder(T,L)}
    \begin{enumerate}
      \item check is current node is \mintinline{prolog}{nil}.
      \item traverse the left subtree \mintinline{prolog}{T1}.
      \item traverse the right subtree \mintinline{prolog}{T2}.
      \item store the node value.
    \end{enumerate}

    \begin{minted}{prolog}
postorder(T,L):-postorder_(T,L),!.
postorder_(nil,[]).
postorder_(bTree(N,T1,T2),List):-
  postorder_(T1,L1),
  postorder_(T2,L2),
  merge_list(Tmp,[N],List),
  merge_list(L1,L2,Tmp),!.
    \end{minted}

  \subsection*{preorder(T,L)}
    \begin{enumerate}
      \item check is current node is \mintinline{prolog}{nil}.
      \item store the node value.
      \item traverse the left subtree \mintinline{prolog}{T1}.
      \item traverse the right subtree \mintinline{prolog}{T2}.
    \end{enumerate}

    \begin{minted}{prolog}
preorder(T,L):-preorder_(T,L),!.
preorder_(nil,[]).
preorder_(bTree(N,T1,T2),[N|T]):-
  preorder_(T1,LT),
  merge_list(LT,RT,T),
  preorder_(T2,RT).
    \end{minted}

  \subsection*{search(T,I)}
    In order to verify that element \mintinline{prolog}{I} belongs to tree
    \mintinline{prolog}{T}, I must traverse the tree, and find a match on
    \mintinline{prolog}{I} with the node value as in
    \mintinline{prolog}{search_(bTree(I,_,_),I):-!.}
    \begin{minted}{prolog}
search(T,I):- search_(T, I),!.
search_(bTree(I,_,_),I):-!.
search_(bTree(_,LL,_),I):-
  search(LL,I).
search_(bTree(_,_,LR),I):-
  search(LR,I).
    \end{minted}

  \pagebreak

  \subsection*{height(T,H)}

    The height of an empty tree is zero, the height of a one-node tree is 1.
    The general case is the height of the tallest subtree. So we can sum 0 and 1s.
    Hence, the height of the tree is \mintinline{prolog}{max(H1,H2,M)} where
    \mintinline{prolog}{H1} is the height of the left subtree,
    \mintinline{prolog}{H2} is the height of the right subtree and
    \mintinline{prolog}{M}, the maximum between those two heights. \\
    I also used an auxiliary function to match the required rule head.

    \begin{minted}{prolog}
height(T,H):-height_(T,H),!.
height_(nil,0).
height_(bTree(nil,_,_),0):-!.
height_(bTree(_,nil,nil),1):-!.
height_(bTree(_,LL,LR),H):-
  height_(LL,H1),
  height_(LR,H2),
  max(H1,H2,M),
  H is 1 + M.
    \end{minted}

  \subsection*{insert(I,T1,T2)}

    \mintinline{prolog}{T2} is the binary tree \mintinline{prolog}{T1 + I}.
    We should find the right node of \mintinline{prolog}{T1} where to append a
    new node containing \mintinline{prolog}{I}. We can compare the value to be
    inserted with the value of the current node and go as far as possible. When
    we'll reach this position, we can insert the new node \mintinline{prolog}{bTree(I,nil,nil)}.

    \begin{minted}{prolog}
insert(I,T1,T2):-insert_(I,T1,T2),!.
insert_(nil,I,bTree(I,nil,nil)):-!. /* new node */
insert_(I,bTree(I,T1,T2),bTree(I,T1,T2)).
insert_(I,bTree(N,T1,T2),bTree(N,NL,T2)):-
  I @=< N,
  insert_(T1,I,NL). /* insert at left */
insert_(I,bTree(N,T1,T2),bTree(N,T1,NR)):-
  I @> N,
  insert_(T2,I,NR). /* insert at right */
    \end{minted}

    \pagebreak

    Here is a simple trace from a unit test where we insert a left node (again,
    I used an auxiliary function so there is \mintinline{prolog}{insert} and
    \mintinline{prolog}{insert_} duplicated lines).

    \begin{myminted}
    ?- trace, insert(2, bTree(5,nil,nil), bTree(5, bTree(2, nil, nil), nil)).
   Call: (9) insert(2, bTree(5, nil, nil), bTree(5, bTree(2, nil, nil), nil)) ? creep
   Call: (10) insert_(2, bTree(5, nil, nil), bTree(5, bTree(2, nil, nil), nil)) ? creep
   Call: (11) 2@=<5 ? creep
   Exit: (11) 2@=<5 ? creep
   Call: (11) insert_(nil, 2, bTree(2, nil, nil)) ? creep
   Exit: (11) insert_(nil, 2, bTree(2, nil, nil)) ? creep
   Exit: (10) insert_(2, bTree(5, nil, nil), bTree(5, bTree(2, nil, nil), nil)) ? creep
   Exit: (9) insert(2, bTree(5, nil, nil), bTree(5, bTree(2, nil, nil), nil)) ? creep
true.
    \end{myminted}

    We have \mintinline{prolog}{2@<=5}, so we insert at left using \mintinline{prolog}{T1}
    on the insert : \mintinline{prolog}{insert_(T1,I,NL).}. This leads us to the
    last node (nil T1) on rule \mintinline{prolog}{insert_(nil,I,bTree(I,nil,nil)):-!.} where
    we cut on applying the new node at \mintinline{prolog}{T2}'s place.

\section*{Exercices}

  This archive contains several unit tests. I used simple test-driven development as a
  software process development. I ensure the unit is small and readable so it can also
  be used as documentation. I found great documentation on
  \footnote{http://www.swi-prolog.org/pldoc/doc\_for?object=section('packages/plunit.html')}.
  I added ascii art tree schema on tests file to document it. The file run\_tests.sh
  is a shortcut to the two prolog commands \mintinline{prolog}{load_test_files([]).} and \mintinline{prolog}{run_tests.}

\end{document}
